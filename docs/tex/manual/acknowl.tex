% Acknowledgements for PETSc Users Manual
%
%   this information is a DUPLICATE of misc/acknwldg.htm
%                MAKE SURE THEY MATCH!!!
%
\noindent {\bf Acknowledgments:}

\medskip \medskip \noindent
We thank all PETSc users for their many suggestions, bug reports, and
encouragement.  We especially thank Victor Eijkhout and David Keyes
for their valuable comments on the source code,
functionality, and documentation for PETSc.


\vspace{.3in}
\noindent
Some of the source code and utilities in PETSc
have been written by 
\begin{itemize}
  \item Mark Adams - scalability features of MPIBAIJ matrices;
  \item Allison Baker - the flexible GMRES code;
  \item Tony Caola - the SPARSEKIT2 ilutp() interface;
  \item Chad Carroll - Win32 graphics;
  \item Cameron Cooper - portions of the VecScatter routines;
  \item Victor Eijkhout;
  \item Paulo Goldfeld - balancing Neumann-Neumann preconditioner;
  \item Matt Hille;
  \item Domenico Lahaye - the interface to John Ruge and Klaus Stueben's AMG;
  \item Peter Mell - portions of the DA routines;
  \item Todd Munson - the LUSOL (sparse solver in MINOS) interface;
  \item Adam Powell - the PETSc Debian package, 
  \item Robert Scheichl - the MINRES implementation,
  \item Liyang Xu - the interface to PVODE;
\end{itemize}

\vspace{.3in}
\noindent
PETSc uses routines from 
\begin{itemize}
  \item BLAS;
  \item LAPACK;
  \item LINPACK -    dense matrix factorization and solve; converted to C using {\tt f2c} and then 
                     hand-optimized for small matrix sizes, for block matrix data structures;
  \item MINPACK -    see page \pageref{sec_fdmatrix}, sequential matrix coloring routines for finite difference Jacobian
                     evaluations; converted to C using {\tt f2c};
  \item SPARSPAK -   see page \pageref{sec_factorization}, matrix reordering routines, converted to C using {\tt f2c};
  \item SPARSEKIT2 - see page \pageref{sec_ilu_icc}, written by Yousef Saad, iludtp(), converted to C using {\tt f2c};
                     These routines 
                     are copyrighted by Saad under the GNU copyright, see \trl{${PETSC_DIR}/src/mat/impls/aij/seq/ilut.c}.
  \item libtfs     - the efficient, parallel direct solver developed by Henry Tufo and Paul Fischer for the direct solution of a coarse grid problem (a linear system with very few degrees of freedom per processor).
\end{itemize}


\vspace{.3in}
\noindent
PETSc interfaces to the following external software:
\begin{itemize}
  \item ADIC/ADIFOR -  automatic differentiation for the computation of sparse Jacobians, 
                     \trllink{http://www.mcs.anl.gov/adic}{http://www.mcs.anl.gov/adic},
                     \trllink{http://www.mcs.anl.gov/adifor}{http://www.mcs.anl.gov/adifor},
  \item AMG -         the algebraic multigrid code of John Ruge and Klaus Stueben,
                     \trllink{http://www.mgnet.org/mgnet-codes-gmd.html}{http://www.mgnet.org/mgnet-codes-gmd.html}
  \item BlockSolve95 - see page \pageref{sec_blocksolve}, for parallel ICC(0) and ILU(0) preconditioning,
                     \trllink{http://www.mcs.anl.gov/blocksolve}{http://www.mcs.anl.gov/blocksolve},
  \item DSCPACK -    see page \pageref{sec_externalsol}, Domain-Separator Codes for solving sparse symmetric
                      positive-definite systems, 
                     developed by Padma Raghavan,   
                     \trllink{http://www.cse.psu.edu/~raghavan/Dscpack/}{http://www.cse.psu.edu/~raghavan/Dscpack/},
  \item ESSL -         IBM's math library for fast sparse direct LU factorization,
  \item Euclid  -   parallel ILU(k) developed by David Hysom, accessed through the Hypre interface,
  \item Hypre -    the LLNL preconditioner library, \trllink{http://www.llnl.gov/CASC/hypre}{http://www.llnl.gov/CASC/hypre}
  \item LUSOL -       sparse LU factorization code (part of MINOS) developed by Michael Saunders,
                      Systems Optimization Laboratory, Stanford University,
                     \trllink{http://www.sbsi-sol-optimize.com/}{http://www.sbsi-sol-optimize.com/},
  \item Mathematica -  see page \pageref{ch_mathematica},
  \item Matlab -      see page \pageref{ch_matlab},
  \item ParMeTiS -     see page \pageref{sec_partitioning}, parallel graph partitioner,
                     \trllink{http://www-users.cs.umn.edu/~karypis/metis/}{http://www-users.cs.umn.edu/~karypis/metis/},
  \item PVODE -       see page \pageref{sec_pvode}, parallel ODE integrator,
                     \trllink{http://www.llnl.gov/CASC/PVODE}{http://www.llnl.gov/CASC/PVODE},
  \item SPAI -        for parallel sparse approximate inverse preconditiong, 
                     \trllink{http://www.sam.math.ethz.ch/~grote/spai/}{http://www.sam.math.ethz.ch/~grote/spai/},
  \item SPOOLES - see page \pageref{sec_externalsol}, SParse Object Oriented Linear Equations Solver, developed by Cleve Ashcraft, 
                    \trllink{http://www.netlib.org/linalg/spooles/spooles.2.2.html}{http://www.netlib.org/linalg/spooles/spooles.2.2.html},
  \item SuperLU and SuperLU\_Dist - see page \pageref{sec_externalsol}, 
                    the efficient sparse LU codes developed by Jim Demmel,  Xiaoye S. Li, and John Gilbert, 
                    \trllink{http://www.nersc.gov/~xiaoye/SuperLU}{http://www.nersc.gov/~xiaoye/SuperLU}.
\end{itemize}
These are all optional packages and do not need to be installed to use PETSc.

PETSc software is developed and maintained with 
\begin{itemize}
\item Bitkeeper revision control system
\item Emacs editor
\end{itemize}

PETSc documentation has been generated using
\begin{itemize}
\item the text processing tools developed by Bill Gropp
\item c2html
\item Microsoft Frontpage
\item pdflatex
\item python
\end{itemize}

