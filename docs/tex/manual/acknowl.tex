% Acknowledgements for PETSc Users Manual
%
% These are also listed on the PETSc homepage, so if you add something here
% add it to the home page also
%
\noindent {\bf Acknowledgments:}

\medskip \medskip \noindent
We thank all PETSc users for their many suggestions, bug reports, and
encouragement.  We especially thank Victor Eijkhout and David Keyes
for their valuable comments on the source code,
functionality, and documentation for PETSc.


\vspace{.3in}
\noindent
Some of the source code and utilities in PETSc (or software used by PETSc)
have been written by 
\begin{itemize}
  \item Mark Adams, scalability features of MPIBAIJ matrices,
  \item Allison Baker, the flexible GMRES code,
  \item Tony Caola, the SPARSEKIT2 ilutp() interface,
  \item Chad Carroll, Win32 graphics,
  \item Cameron Cooper, portions of the VecScatter routines, 
  \item Victor Eijkhout, KSP type BICG, VecPipeline() and VecXXXBegin()/End() routines, 
  \item Paulo Goldfeld, balancing Neumann-Neumann preconditioner,
  \item Matt Hille, 
  \item Domenico Lahaye, the interface to John Ruge and Klaus Stueben's AMG,
  \item Peter Mell, portions of the DA routines,
  \item Todd Munson, the LUSOL (sparse solver in MINOS) interface,
  \item Wing-Lok Wan, the ILU portion of BlockSolve95,
  \item Liyang Xu, the interface to PVODE.
\end{itemize}

\vspace{.3in}
\noindent
PETSc uses routines from 
\begin{itemize}
  \item BLAS
  \item LAPACK
  \item LINPACK      matrix factorization and solve; converted to C using {\tt f2c} and then 
                      hand-optimized for small matrix sizes, for block matrix data structures,
  \item MINPACK      sequential matrix coloring routines for finite difference Jacobian
                       evaluations; converted to C using {\tt f2c},
  \item SPARSPAK     matrix reordering routines, converted to C using {\tt f2c},
  \item SPARSEKIT2   written by Yousef Saad, iludtp(), converted to C using {\tt f2c}. These routines 
                     are copyrighted by Saad under the GNU copyright, see \trl{${PETSC_DIR}/src/mat/impls/aij/seq/ilut.c}.
  \item libtfs the efficient, parallel direct solver developed by Henry Tufo and Paul Fischer.
\end{itemize}


\vspace{.3in}
\noindent
PETSc interfaces to the following external software:
\begin{itemize}
  \item AMG          the algebraic multigrid code of John Ruge and Klaus Stueben,
                     \trllink{http://www.mgnet.org/mgnet-codes-gmd.html}{http://www.mgnet.org/mgnet-codes-gmd.html}
  \item BlockSolve95 for parallel ICC(0) and ILU(0) preconditioning,
                     \trllink{http://www.mcs.anl.gov/blocksolve}{http://www.mcs.anl.gov/blocksolve},
  \item ESSL         IBM's math library for fast sparse direct LU factorization,
  \item LUSOL        sparse LU factorization code (part of MINOS) developed by Michael Saunders,
                      Systems Optimization Laboratory, Stanford University,
                     \trllink{http://www.sbsi-sol-optimize.com/}{http://www.sbsi-sol-optimize.com/},
  \item Matlab       
  \item ParMeTiS      parallel graph partitioner, \trllink{http://www-users.cs.umn.edu/~karypis/metis/}{http://www-users.cs.umn.edu/~karypis/metis/},
  \item PVODE        parallel ODE integrator, \trllink{http://www.llnl.gov/CASC/PVODE}{http://www.llnl.gov/CASC/PVODE},
  \item SPAI         for parallel sparse approximate inverse preconditiong, 
                     \trllink{http://www.sam.math.ethz.ch/~grote/spai/}{http://www.sam.math.ethz.ch/~grote/spai/}.
\end{itemize}
These are all optional packages and do not need to be installed to use PETSc.


