	The purpose of this document is to outline the current strategic thinking for development across the PETSc
group, including auxillary projects, and perhaps overlying projects such as TAO or SLEPc. The main thrusts should
include not only technical justifications, but application partnerships and actual/possible funding sources. We will
organize these thrusts into sections.

\section{Definitions}

   {\bf Stateless iterative solver} - An iterative process that, for each iteration, the algorithm depends {\bf only}
on the current initial solution and the problem. The PCs in PETSc are stateless. The ``plain'' Newton's method in
PETSc is stateless.

   {\bf State iterative solver} - An iterative process where the algorithm depends directly on not only the current
initial solution but additional information from previous iterations. The KSP methods in PETSc are state. Newton's method
with lagged Jacobians or the E-W convergence test are state. Nonlinear CG and nonlinear GMES (and minimization methods
such as l-BFGS) are state iterative solvers.

   {\bf Stateless iterative solver implementation} - An implementation of a stateless iterative solver that is also 
stateless. Almost no implementation in PETSc is stateless, for example, the ILU PC computes the factorization initially
which is then used for each iteration. Even Jacobi extracts the diagonal initially to use later. There is a generalization
of this where the iterations do not depend directly on previous iterations (except through the initial solution) but
do depend on initial setup (like the construction of the ILU factors), we will call these {\bf setup state only iterative solver
implementations}. Stateless iterative solvers are {\bf only} 
implemented as setup state only for efficiency reasons (to prevent recomputation). 

  We will call stateless iterative solvers {\bf preconditioners} and state iterative solvers {\bf accelerators} or 
possibly Krylov methods or Krylov accelerators. Preconditioners can be viewed as ``simple'' transformations of the initial
(nonlinear) algebraic problem that results in a new algebraic problem where simple iteration (nonlinear Richardson's method)
may converge better. FAS is a preconditioner, nonlinear CG is an accelerator. Newton's method with a direct solver is a preconditioner but is 
funky because the transformation depends not only on the problem but also on the current initial solution; the transformed
problem is $ J^{-1}(x^{n}) F(x^n)$. It is interesting to note that each transformation is linear (like for all linear preconditioners).
Newton's method with an iterative method is an accelerator.


\section{Matrix-Free Operation}

	This is a large project of several interlocking parts. However, at a high level, the goal can be seen as
elimination of parallel sparse matrices from the nonlinear solution process.

	The first step in this process is the elimination of Newton's algorithm as the outer parallel solver for the
nonlinear algebraic problem. Barry has cleaned up the existing FAS code, and introduced a point-block version. A summer
student from Olof Widlund's group at NYU is also working along this line. She will hopefully examine effective
{\bf accelerators}  for FAS. We expect to improve our performance from the dismal 5\% of peak numbers to somewhere around
the STREAMS benchmark with these changes.

	The next step involves better control of the mesh and discretization for common problems. FAS operates on sets
of fully assembled matrix rows, as opposed to the common method of integrating each element at a time. Using FIAT and
FFC, we have the ability to reorganize the computation to allow computation of one dof at a time. Here we also have the
possibility of improving the efficiency yet again with sparser meshes supporting higher order elements.

  \subsection{Parnerships}

	The best test of these improvements is likely to come from the code under development by Ethan Coon at Columbia.

  \subsection{Funding sources}

	Hopefully our multiscale grant comes through. I would say this is also excellent for a followon SciDAC.

\section{Scalability}

	The idea here is to greatly increase the use of PETSc geometric multigrid, which we believe is greatly
underutilized. To this end, we propose integrating several geometric multigrid algorithms currently in use in the
geophysical community as PCMG implementations along the same line as the Sandia ML package.

	I have sent an initial email to VPAC and Monash to initiate development of the first integration project. We
will introduce the current St. Germaine multigrid into PETSc.

  \subsection{Parnerships}

	We will have strong support from CIG, and in particular Louis Moresi from Monash and Steve Quenette from VPAC.
However we can also expect some help from Scott King at Purdue, John Baumgartner at LANL, and perhaps Marc Spiegelman at
Columbia.

  \subsection{Funding sources}

	Initially all this money will be coming from CIG in the form of my support. However, it seems likely that we
will be able to leverage work from the application partners.

\section{Mesh Representation}

	Dmitry and I have developed an extremely flexible and general topology representation which we call a Sieve.
This mechanism will be integrated into the next version of PETSc. It provides a parallel unstructured mesh interface
which will mirror DMDA, and allow a DMMG-like object to painlessly introduce scalable preconditioning into unstructured
simulations like PCICE and PyLith.

  \subsection{Parnerships}

	We plan to use this initially in the PCICE code from Rich Martineau at INL, and PyLith from Charles Williams at
RPI.

  \subsection{Funding sources}

	Right now, Rich Martineau is using some of his LDRD money for this. Dmitry hopes to receive funding for this in
two recently submitted grants.
